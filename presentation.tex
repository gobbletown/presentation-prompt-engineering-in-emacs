% Created 2021-03-02 Tue 11:16
% Intended LaTeX compiler: pdflatex
\documentclass[presentation]{beamer}
\usepackage[utf8]{inputenc}
\usepackage[T1]{fontenc}
\usepackage{graphicx}
\usepackage{grffile}
\usepackage{longtable}
\usepackage{wrapfig}
\usepackage{rotating}
\usepackage[normalem]{ulem}
\usepackage{amsmath}
\usepackage{textcomp}
\usepackage{amssymb}
\usepackage{capt-of}
\usepackage{hyperref}
\usetheme[height=20pt]{Rochester}
\author{Shane Mulligan \\  }
\date{\textit{<2021-03-01 Mon>}}
\title{Presenting\ldots{} \\   \emph{\alert{Prompt Engineering in Emacs}} \\  }
\hypersetup{
 pdfauthor={Shane Mulligan \\  },
 pdftitle={Presenting\ldots{} \\   \emph{\alert{Prompt Engineering in Emacs}} \\  },
 pdfkeywords={},
 pdfsubject={},
 pdfcreator={Emacs 27.0.91 (Org mode 9.3)}, 
 pdflang={English}}
\begin{document}

\maketitle

\section{Presentation}
\label{sec:orgf083607}
\begin{frame}[label={sec:org2df968d}]{Repositories for following along}
{\footnotesize
\begin{center}
\begin{tabular}{l}
\url{http://github.com/mullikine/presentation-prompt-engineering-in-emacs}\\
\url{http://github.com/semiosis/examplary}\\
\url{http://github.com/semiosis/pen.el}\\
\url{http://github.com/semiosis/prompts}\\
\url{http://github.com/semiosis/prompt-engineering-patterns}\\
\end{tabular}
\end{center}
}
\end{frame}

\section{Preliminaries}
\label{sec:org629ed9f}
\subsection{Background Knowledge}
\label{sec:org1065c2b}
\begin{itemize}
\item \texttt{GPT-3} is a \texttt{seq2seq} model
A text generator.
\begin{itemize}
\item It's stochastic
but configurable to be deterministic.
\end{itemize}
\end{itemize}

\begin{frame}[label={sec:org0169f84}]{Key concepts}
\begin{itemize}
\item prompt,
\item completion, and
\item tokens
\end{itemize}
\end{frame}

\begin{frame}[label={sec:org7448e54}]{Limitation}
Combined, the text prompt and generated
completion must be below 2048 tokens (roughly
\textasciitilde{}1500 words).
\end{frame}

\begin{frame}[label={sec:org36a9117}]{Search engine vs Database}
\begin{itemize}
\item \uline{Relational Databases} use a \uline{B-Tree index}.
\item \alert{Search engines} mostly use \alert{inverted index}.q
\item \uline{Relational Databases} give you what you \uline{asked for}.
\item \alert{Search engines} give you what you \alert{wanted}.
\end{itemize}
\end{frame}

\begin{frame}[label={sec:org914b2c6}]{Terminology}
\begin{description}
\item[{\uline{Indices}}] = indexes. Indexes just sounds wrong to me.
\item[{\uline{Model}}] The \alert{set of functions} that describe the relations between variables.
\end{description}

\begin{quote}
"Probabilistic and information theoretic methods are used to make results better anyway.
Compromises are made anyway. Query reformulation, drift, etc.
So it is just a natural progression to use NNs for some of these components? Am I right." -- A quote from myself.
\end{quote}
\end{frame}
\end{document}
