% Created 2021-03-01 Mon 21:57
% Intended LaTeX compiler: pdflatex
\documentclass[presentation]{beamer}
\usepackage[utf8x]{inputenc}
\usepackage[T1]{fontenc}
\usepackage{graphicx}
\usepackage{grffile}
\usepackage{longtable}
\usepackage{wrapfig}
\usepackage{rotating}
\usepackage[normalem]{ulem}
\usepackage{amsmath}
\usepackage{textcomp}
\usepackage{amssymb}
\usepackage{capt-of}
\usepackage{hyperref}
\usetheme{default}
\author{Shane Mulligan \\  }
\date{\textit{<2021-03-01 Mon>}}
\title{presenting\ldots{} \\   \emph{\alert{Prompt Engineering in Emacs}} \\  }
\hypersetup{
 pdfauthor={Shane Mulligan \\  },
 pdftitle={presenting\ldots{} \\   \emph{\alert{Prompt Engineering in Emacs}} \\  },
 pdfkeywords={},
 pdfsubject={},
 pdfcreator={Emacs 27.0.91 (Org mode 9.3.6)}, 
 pdflang={English}}
\begin{document}

\maketitle

\section{Presentation}
\label{sec:org67790be}
\subsection{"\emph{Prompt Engineering in Emacs}"}
\label{sec:org0ccc55f}
\begin{center}
\begin{tabular}{ll}
Speaker & Shane Mulligan\\
\end{tabular}
\end{center}

\begin{frame}[label={sec:org299757d}]{Repositories to follow along}
\begin{center}
\begin{tabular}{ll}
Slides & \url{http://github.com/mullikine/presentation-prompt-engineering-in-emacs}\\
DSL & \url{http://github.com/semiosis/examplary}\\
emacs & \url{http://github.com/semiosis/pen.el}\\
prompts & \url{http://github.com/semiosis/prompts}\\
 & \url{http://github.com/mullikine/prompt-engineering-patterns}\\
\end{tabular}
\end{center}
\end{frame}

\section{Preliminaries}
\label{sec:org44d37b2}
\subsection{Background Knowledge}
\label{sec:org04355f3}
\begin{description}
\item[{Deep learning models}] are function approximators.
\end{description}

\begin{frame}[label={sec:org3efe7c5}]{Search engine vs Database}
\begin{itemize}
\item \uline{Relational Databases} use a \uline{B-Tree index}.
\item \alert{Search engines} mostly use \alert{inverted index}.q
\item \uline{Relational Databases} give you what you \uline{asked for}.
\item \alert{Search engines} give you what you \alert{wanted}.
\end{itemize}
\end{frame}

\begin{frame}[label={sec:org3037c1c}]{Terminology}
\begin{description}
\item[{\uline{Indices}}] = indexes. Indexes just sounds wrong to me.
\item[{\uline{Model}}] The \alert{set of functions} that describe the relations between variables.
\end{description}

\begin{quote}
"Probabilistic and information theoretic methods are used to make results better anyway.
Compromises are made anyway. Query reformulation, drift, etc.
So it is just a natural progression to use NNs for some of these components? Am I right." -- A quote from myself.
\end{quote}
\end{frame}
\end{document}